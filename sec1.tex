\section{財政計算関係}

\subsection{財政計算関係}

\subsubsection{リスク対応額(年金2 2018 問題3(2)の類題)}


\begin{itembox}[l]{\textgt{ポイント}}
  
  リスク対応額の計算問題で調べることは以下の通り。

  \begin{enumerate}
    \item \red{リスク対応掛金を特別掛金に振り替えることができるか。}\\
    (特別掛金収入現価の増加分以内でリスク対応額を減らすことができる)\\
    (則46条の2第2項第1号)
    \item \red{財源が確保されていない部分は増加するか。}\\
    (「財政悪化リスク相当額$-$リスク充足額」が直近のリスク対応掛金収入現価算定時から増える分だけリスク対応額を増やすことができる)\\
    (則46条の2第2項第3号)
    \item \red{リスク充足額が財政悪化リスク相当額以下であるか。}\\
    (リスク充足額が財政悪化リスク相当額を上回っている場合、その上回っている額だけリスク対応額を減らさなければならない)\\
    (則46条の2第3項)
  \end{enumerate}



  リスク充足額は、
  「$Max
  (積立金 + 標準掛金収入現価+ 特別掛金収入現価+リスク対応掛金収入現価
  - 通常予測給付現価, 0)$」
  で計算する。

\end{itembox}

\begin{sol}
  \;

  \textbf{<A社について>}

  (下限)

  特別掛金収入現価の増加分を調べる。(特に指定がない限り、数値は変更後のものを用いる)
  \begin{flalign*}
    & 財政再計算後の過去勤務債務& \\
    &= 数理債務 - (積立金 - 別途積立金) \\
    &= (通常予測給付現価 - 標準掛金収入現価) - (積立金 - 別途積立金)& \\
    &= (18,810 - 6,710)- (8,910 - 880)& \\
    &= 4,070&
  \end{flalign*}
  \begin{flalign*}
    & 特別掛金収入現価の増加分& \\
    &= 4,070 -1,760&\\
    &= 2,310&
  \end{flalign*}
  したがって、リスク対応掛金を$2,310$(百万円)まで減らすことができる。
  再計算前のリスク対応掛金収入現価は$1,210$(百万円)なので、
  リスク対応額の下限は\red{0円}。

  \;

  (上限)

  再計算後の財源が確保されていない部分を計算する。
  \begin{flalign*}
    & 財政悪化リスク相当額 - リスク充足額& \\
    & = 財政悪化リスク相当額 - 
    (積立金 + 標準掛金収入現価+ 特別掛金収入現価+リスク対応掛金収入現価(変更前)& \\
    &- 通常予測給付現価)& \\
    & = 2,860 - Max(8,910 + 6,710 + 4,070 + 1,210 - 18,810, 0)& \\
    & = 770&
  \end{flalign*}

  また、再計算前のリスク充足額は、
  「※直前の財政再計算にてリスク対応額の上限額に基づき、初めて設定している」
  という記述から、0円であることがわかる。

  よって、直前のリスク対応掛金額$1,210$百万円に
  リスク充足額の増加分$(770-0)$百万円を加算できるので、
  \red{$1,980$百万円}がリスク対応額の上限となる。

  \;

  \textbf{<B社について>}

  特別掛金収入現価の増加分を調べる。
  \begin{flalign*}
    & 財政再計算後の過去勤務債務& \\
    &= (20,020 - 5,720)- (13,530 - 1,320)& \\
    &= 2,090&
  \end{flalign*}
  \begin{flalign*}
    & 特別掛金収入現価の増加分& \\
    &= 2,090 - 2,530&\\
    &< 0&
  \end{flalign*}
  よって、特別掛金収入現価が再計算前後で減少しているため、
  リスク対応額は特別掛金に振り替えることはできない。


  また、再計算後の財源が確保されていない部分を計算すると、
  \begin{flalign*}
    & 財政悪化リスク相当額 - リスク充足額& \\
    & = 2,970 - Max(13,530 + 5,720 + 2,090 + 2,530 - 20,020, 0)& \\
    & = -880&
  \end{flalign*}
  $財政悪化リスク相当額 < リスク充足額$となっているので、
  リスク対応額を$財政悪化リスク相当額 - リスク充足額$だけ減らす必要がある。
  よって、リスク対応額の上限および下限は$2,530 - 880 = \red{1,650}$\red{百万円}。


\end{sol}

\newpage

